\nonstopmode{}
\documentclass[a4paper]{book}
\usepackage[times,inconsolata,hyper]{Rd}
\usepackage{makeidx}
\usepackage[utf8,latin1]{inputenc}
% \usepackage{graphicx} % @USE GRAPHICX@
\makeindex{}
\begin{document}
\chapter*{}
\begin{center}
{\textbf{\huge Package `userinput'}}
\par\bigskip{\large \today}
\end{center}
\begin{description}
\raggedright{}
\item[Type]\AsIs{Package}
\item[Title]\AsIs{Validating Command Line User Input}
\item[Version]\AsIs{0.1.0}
\item[Author]\AsIs{Philip Eichinski }\email{philip.eichinski@qut.edu.au}\AsIs{}
\item[Maintainer]\AsIs{Philip Eichinski }\email{philip.eichinski@qut.edu.au}\AsIs{}
\item[Description]\AsIs{Provides functions for common user input situations, like confirm,
select from a list of values etc.}
\item[License]\AsIs{GPL-2}
\item[LazyData]\AsIs{TRUE}
\item[RoxygenNote]\AsIs{5.0.1}
\end{description}
\Rdcontents{\R{} topics documented:}
\inputencoding{utf8}
\HeaderA{Confirm}{Prompts the user for confirmation in the console}{Confirm}
%
\begin{Description}\relax
Prompts the user for confirmation in the console
\end{Description}
%
\begin{Usage}
\begin{verbatim}
Confirm(msg, default = NULL)
\end{verbatim}
\end{Usage}
%
\begin{Arguments}
\begin{ldescription}
\item[\code{msg}] string the thing that you are want a yes/no answer to

\item[\code{default}] mixed optional. If supplied and y or yes or similar, will default to true.
If supplied and not similar to yes will default to no.
If not supplied, will not have a default and the user must choose an option
\end{ldescription}
\end{Arguments}
%
\begin{Details}\relax
Presents the user with a yes/no question and returns true or false depending on their answer
\end{Details}
%
\begin{Value}
boolean
\end{Value}
\inputencoding{utf8}
\HeaderA{GetDirectory}{prompts the user for a directory}{GetDirectory}
%
\begin{Description}\relax
prompts the user for a directory
\end{Description}
%
\begin{Usage}
\begin{verbatim}
GetDirectory(msg = "please enter a path to the directory",
  create.if.missing = FALSE)
\end{verbatim}
\end{Usage}
%
\begin{Arguments}
\begin{ldescription}
\item[\code{msg}] the prompt to show to the user

\item[\code{create.if.missing}] boolean whether to create the directory if it is missing or prompt
\end{ldescription}
\end{Arguments}
%
\begin{Details}\relax
after the user enters in a directory, it will check if the directory exists.
If it doesn't, it will prompt the user if they want to create it, or if create.if.missing is TRUE
will create it without asking. It will only create the directory itself, not parent directories.
e.g. if the user enters /a/b/c and /a/b doesn't exist, it will not create it. But if /a/b exists and
/a/b/c doesn't exist, it will prompt to create c
@export
\end{Details}
\inputencoding{utf8}
\HeaderA{GetMultiUserchoice}{allows the user to select 1 or more of the choices,}{GetMultiUserchoice}
%
\begin{Description}\relax
allows the user to select 1 or more of the choices,
\end{Description}
%
\begin{Usage}
\begin{verbatim}
GetMultiUserchoice(options, choosing.what = "one of the following",
  default = 1, all = FALSE)
\end{verbatim}
\end{Usage}
%
\begin{Arguments}
\begin{ldescription}
\item[\code{options}] string vector; list of choices

\item[\code{choosing.what}] string; instrucitons for user

\item[\code{default}] int or string "all"; which options should be selected if the just hits clicks 'enter'

\item[\code{all}] boolean; should there be an extra option at the end to choose all the options in the list?
\end{ldescription}
\end{Arguments}
%
\begin{Value}
int vector of the choice numbers
\end{Value}
\inputencoding{utf8}
\HeaderA{GetPresets}{Returns the preset input}{GetPresets}
%
\begin{Description}\relax
Returns the preset input
\end{Description}
%
\begin{Usage}
\begin{verbatim}
GetPresets()
\end{verbatim}
\end{Usage}
%
\begin{Value}
character
\end{Value}
\inputencoding{utf8}
\HeaderA{GetUserChoice}{Prompts the user to choose one of the given set of choices}{GetUserChoice}
%
\begin{Description}\relax
Prompts the user to choose one of the given set of choices
\end{Description}
%
\begin{Usage}
\begin{verbatim}
GetUserChoice(choices, choosing.what = "one of the following", default = 1,
  allow.range = FALSE, optional = FALSE)
\end{verbatim}
\end{Usage}
%
\begin{Arguments}
\begin{ldescription}
\item[\code{choices}] vector of strings

\item[\code{choosing.what}] string; used for presenting the instructions to the user

\item[\code{default}] int if the user just hits enter, this will be chosen

\item[\code{allow.range}] boolean if TRUE, the user can enter something like 2-4 which will return c(2,3,4)

\item[\code{optional}] boolean if TRUE, user can select 0 to return false (i.e. no choice)
\end{ldescription}
\end{Arguments}
%
\begin{Value}
int the index of the choice selected by the user
\end{Value}
\inputencoding{utf8}
\HeaderA{Preset}{sets the preset input global variable, which if not empty will be used instead of .ReadLine.}{Preset}
%
\begin{Description}\relax
Allows tests to preset the userinput without interrupting the test with .ReadLine.
\end{Description}
%
\begin{Usage}
\begin{verbatim}
Preset(user.input.strings = character())
\end{verbatim}
\end{Usage}
%
\begin{Arguments}
\begin{ldescription}
\item[\code{user.input.strings}] character
\end{ldescription}
\end{Arguments}
%
\begin{Details}\relax
This should probably not be used except for its designed purpose of unit
tests on scripts that use userinput. It could cause unexpected behaviour if
by mistake something is left in the preset.input variable. Use on.exit(Preset())
\end{Details}
\inputencoding{utf8}
\HeaderA{ReadInt}{Reads an int input from the user and re-prompts if they didn't enter an int}{ReadInt}
%
\begin{Description}\relax
Reads an int input from the user and re-prompts if they didn't enter an int
\end{Description}
%
\begin{Usage}
\begin{verbatim}
ReadInt(msg = "Enter an integer", min = 1, max = NA, default = NULL)
\end{verbatim}
\end{Usage}
%
\begin{Arguments}
\begin{ldescription}
\item[\code{msg}] character

\item[\code{min}] int

\item[\code{max}] int

\item[\code{default}] int optional
\end{ldescription}
\end{Arguments}
\printindex{}
\end{document}
